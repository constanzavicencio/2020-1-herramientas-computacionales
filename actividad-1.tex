\documentclass{article}
\usepackage[utf8]{inputenc}
\usepackage{graphicx} %

\begin{document}

\section{CINEMÁTICA III}
\underline{\textbf{PROBLEMA 1:}} En la figura se ilustra una sala rectangular de longitud \underline{b} y ancho \underline{$\lambda b$}. Desde la esquina $P$ será arrastrada en forma rectílinea una zanahoria con rapidez constante \underline{$u$} para desaparecer por la esquina $Q$. Desde el rincón $E$ correrá en forma recta una liebre con rapidez constante \underline{$v$} para alcanzar a la zanahoria $(v>u)$. La liebre se propone atrapar la zanahoria a punto de desaparecer por $Q$.

[4Pt] Determine el lapso $T$ que debe esperar la liebre para comenzar su carrera a contar del instante en que la zanahoria emerge por $P$.

[2Pt] Determine el ancho mínimo de la sala $\lambda p$ que permite que la liebre atrape la zanahoria antes de desaparecer por la esquina $Q$.

\begin{figure}[h!]
\centering
\includegraphics[width=0.3\textwidth]{Imagen 1.jpg}
\end{figure}

\section{Solución}

\begin{figure}[h!]
\centering
\includegraphics[width=0.3\textwidth]{Imagen 2.jpg}
\end{figure}

Sea $t$ el lapso que tarda la liebre en llegar a la esquina $Q$. Entonces, de acuerdo al triángulo dibujado se tiene

$$(vt^2)=b^{2}+\lambda b^{2} \rightarrow vt=b\sqrt{1+\lambda^{2}}$$

Considerando el cateto superior...

$$uT+ut=\lambda b$$

Combinando ambos resultados y despejando $T$...

$$T=\frac{\lambda b}{u}-\frac{b}{v}\sqrt{1+\lambda ^{2}}$$

Para que la liebre alcance la zanahoria se exige que $T\geq 0$. Entonces

$$\frac{\lambda b}{u}\geq \frac{vb}{v}\sqrt{1+\lambda^{2}}\rightarrow \lambda^{2}\left(\frac{1}{u^{2}}-\frac{1}{v^{2}}\right)\geq \frac{1}{v^{2}}$$
con lo cual
$$\lambda b \geq b \frac{u}{\sqrt{v^{2}-u^{2}}}$$


\end{document}
